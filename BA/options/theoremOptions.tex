% theoremOptions.tex

% Theorem-Optionen %
\theoremseparator{.}
\theoremstyle{change}
\newtheorem{theorem}{Theorem}[section]
\newtheorem{satz}[theorem]{Satz}
\newtheorem{lemma}[theorem]{Lemma}
\newtheorem{korollar}[theorem]{Korollar}
\newtheorem{proposition}[theorem]{Proposition}
% Ohne Numerierung
\theoremstyle{nonumberplain}
\renewtheorem{theorem*}{Theorem}
\renewtheorem{satz*}{Satz}
\renewtheorem{lemma*}{Lemma}
\renewtheorem{korollar*}{Korollar}
\renewtheorem{proposition*}{Proposition}
% Definitionen mit \upshape
\theorembodyfont{\upshape}
\theoremstyle{change}
\newtheorem{definition}[theorem]{Definition}
\theoremstyle{nonumberplain}
\renewtheorem{definition*}{Definition}
% Kursive Schrift
\theoremheaderfont{\itshape}
\newtheorem{notation}{Notation}
\newtheorem{konvention}{Konvention}
\newtheorem{bezeichnung}{Bezeichnung}
\theoremsymbol{\ensuremath{\Box}}
\newtheorem{beweis}{Beweis}
\theoremsymbol{}
\theoremstyle{change}
\theoremheaderfont{\bfseries}
\newtheorem{bemerkung}[theorem]{Bemerkung}
\newtheorem{beobachtung}[theorem]{Beobachtung}
\newtheorem{beispiel}[theorem]{Beispiel}
\newtheorem{problem}{Problem}
\theoremstyle{nonumberplain}
\renewtheorem{bemerkung*}{Bemerkung}
\renewtheorem{beispiel*}{Beispiel}
\renewtheorem{problem*}{Problem}
%EOF