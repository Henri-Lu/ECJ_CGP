% erklaerung.tex
\parindent 0pt
\cleardoublepage

{\centering \hspace*{1cm}\Large \textbf{Eidesstattliche Versicherung} 

}

\normalsize
\vspace*{1cm}

\parbox{5cm}{ \centering \MeinNachname, \MeinVorame 
\vspace*{0,1cm}
\hrule
\vspace*{0,1cm}
\strut \small \ Name, Vorname} \hfill
\parbox{3cm}{ \centering \MeineMatrikelnummer 
\vspace*{0,15cm}
\hrule
\vspace*{0,1cm}
\strut \small \ Matr.-nr.}

\vspace*{1cm}

Ich versichere hiermit an Eides statt, dass ich die vorliegende \MeineArbeit mit dem Titel

\vspace{0,25cm}
{
\centering
\textbf{\MeinTitel}

}
\vspace{0,25cm}

selbstständig und ohne unzulässige fremde Hilfe erbracht habe. Ich habe keine anderen als die angegebenen Quellen und Hilfsmittel benutzt sowie wörtliche und sinngemäße Zitate kenntlich gemacht. Die Arbeit hat in gleicher oder ähnlicher Form noch keiner Prüfungsbehörde vorgelegen. 
\vspace*{1cm}

\parbox{6.2cm}{ Dortmund, den \today
\vspace*{0,1cm}
\hrule
\vspace*{0,2cm}
\strut \small \ Ort, Datum} \hfill
\parbox{5cm}{ \color{white} Platzhalter \color{black}
\vspace*{0,1cm}
\hrule
\vspace*{0,2cm}
\strut \small \ Unterschrift}

\vspace*{1,5cm}

\textbf{Belehrung:} \vspace{0,25cm}
\newline
Wer vorsätzlich gegen eine die Täuschung über Prüfungsleistungen betreffende Re\-ge\-lung einer Hochschulprüfungsordnung verstößt, handelt ordnungswidrig. Die Ordnungs\-widrig\-keit kann mit einer Geldbuße von bis zu 50.000,00 \euro\ geahndet werden. Zuständige Verwaltungsbehörde für die Verfolgung und Ahndung von Ordnungswidrigkeiten ist der Kanzler/ die Kanzlerin der Technischen Universität Dortmund. Im Falle eines mehrfachen oder sonstigen schwerwiegenden Täuschungsversuches kann der Prüfling zudem exmatrikuliert werden. (\textsection\ 63 Abs. 5 Hochschulgesetz - HG - )
\vspace{0,25cm} \newline
Die Abgabe einer falschen Versicherung an Eides statt wird mit Freiheitsstrafe bis zu 3 Jahren oder mit Geldstrafe bestraft.
\vspace{0,25cm} \newline
Die Technische Universität Dortmund wird gfls. elektronische Vergleichswerkzeuge (wie z.B. die Software „turnitin“) zur Überprüfung von Ordnungswidrigkeiten in Prüfungsverfahren nutzen.
\vspace{0,25cm} \newline
Die oben stehende Belehrung habe ich zur Kenntnis genommen:
\vspace*{1cm}

\parbox{6.2cm}{ Dortmund, den \today
\vspace*{0,1cm}
\hrule
\vspace*{0,1cm}
\strut \small \ Ort, Datum} \hfill
\parbox{5cm}{ \color{white} Platzhalter \color{black}
\vspace*{0,1cm}
\hrule
\vspace*{0,1cm}
\strut \small \ Unterschrift}

%
%\addcontentsline{toc}{chapter}{Eidesstattliche Versicherung}


%\cleardoublepage
%\textbf{
%Die unten stehende Erklärung ist nur für Diplomarbeiten geeignet.
%Für Bachelor- und Masterarbeiten diese Seite vollständig aus der Arbeit entfernen und stattdessen die vorgebene Erklärung vom Prüfungsamt mit in die Arbeit einbinden.
%Diese muss nicht im Inhaltsverzeichnis auftauchen und muss auch keine Seitenzahl haben.
%Einfach nur das ausgefüllte Blatt ganz am Ende mit in die Arbeit einbinden.
%Die Erklärung findet sich auf der Seite des Prüfungsamts (TU Homepage $\rightarrow$ Studierende $\rightarrow$ Prüfungsangelegenheiten) unter dem Punkt Bachelor- und Masterarbeiten.
%Diesen fett gedruckten Text auf jeden Fall aus der Arbeit entfernen!
%}
%\\
%\normalsize
%Hiermit versichere ich, dass ich die vorliegende Arbeit selbstständig verfasst habe und keine anderen als die angegebenen Quellen und Hilfsmittel verwendet sowie Zitate kenntlich gemacht habe.\\\\
%Dortmund, den \today \\\\\\\\
%Muster Mustermann
% EOF